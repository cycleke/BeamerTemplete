\usepackage[boxed,linesnumbered,lined]{algorithm2e}
\usepackage{amsmath}
\usepackage{babel}
\usepackage{bookmark}
\usepackage{booktabs} %% 这个生成的表格比较好看,里面严禁使用竖线。
\usepackage{colortbl,dcolumn}     %% 彩色表格
\usepackage{environ}
\usepackage{fancyhdr}        %% header footer
\usepackage{graphicx}
\usepackage{hyperref}
\usepackage{listings}
\usepackage{multimedia}
\usepackage{multicol,multienum}
\usepackage{newtxmath}
\usepackage{palatino}
\usepackage{pgf}
\usepackage{pgfpages}
\usepackage{pifont}
\usepackage{siunitx}
\usepackage{subfigure} %%图形或表格并排排列
\usepackage{tabularx}
\usepackage{tikz}
\usepackage{times}
\usepackage{ulem}
\usepackage{varwidth}
\usepackage{xcolor}
\usepackage[BoldFont,SlantFont,CJKchecksingle]{xeCJK}
\usepackage{xltxtra}
\usepackage{xunicode}

\usetikzlibrary{chains}

\graphicspath{{figures/}}         %% 图片路径. 本文的图片都放在这个文件夹里了.
\allowdisplaybreaks{}
\setlength{\parindent}{2em}

\makeatletter
\newcommand{\rmnum}[1]{\romannumeral#1}
\newcommand{\Rmnum}[1]{\expandafter\@slowromancap\romannumeral#1@}

\usefonttheme[onlymath]{serif}
\setCJKfamilyfont{hei}{SimHei}
\newcommand{\hei}{\CJKfamily{hei}}
\setCJKfamilyfont{siyuan}{SourceHanSerifCN-Regular}
\newcommand{\sy}{\CJKfamily{siyuan}}
\defaultfontfeatures{Mapping=tex-text}
\setCJKmainfont[BoldFont=SourceHanSansCN-Bold]{SourceHanSerifCN-Regular}
\setCJKsansfont[BoldFont=SourceHanSansCN-Bold]{SourceHanSerifCN-Regular}
\setCJKmonofont[BoldFont=SourceHanSansCN-Bold]{SourceHanSerifCN-Regular}
\setsansfont{SF Pro Display}
\setmainfont{Serif UI Display}

\definecolor{mygreen}{rgb}{0,0.6,0}
\definecolor{mygray}{rgb}{0.5,0.5,0.5}
\definecolor{mymauve}{rgb}{0.58,0,0.82}
\newcommand{\Console}{Console}
\lstset{
  % backgroundcolor=\color{white},   % choose the background color
  basicstyle=\footnotesize\ttfamily, % size of fonts used for the code
  columns=fullflexible,
  breaklines=true,                 % automatic line breaking only at whitespace
  captionpos=b,                    % sets the caption-position to bottom
  tabsize=2,
  commentstyle=\color{mygreen},    % comment style
  escapeinside={\%*}{*\)},          % if you want to add LaTeX within your code
  keywordstyle=\color{blue},       % keyword style
  stringstyle=\color{mymauve}\ttfamily,     % string literal style
  % numbers=left,
  frame=single,
  rulesepcolor=\color{red!20!green!20!blue!20},
  % identifierstyle=\color{red}
}

\definecolor{color1}{rgb}{0.22,0.45,0.70}  % light blue
\definecolor{color2}{rgb}{0.45,0.45,0.45}  % dark grey

%% 进度条
\newcommand{\progressbar}[2][2cm]{%
  \textcolor{color1}{\rule{#1 * \real{#2} / 100}{1.5ex}}%
  \textcolor{color2!15}{\rule{#1-#1 * \real{#2} / 100}{1.5ex}}}

\hypersetup{
  colorlinks=true,
  linkcolor=blue,
  filecolor=magenta,
  urlcolor=cyan,
}


\setbeamertemplate{theorems}[numbered]
\newtheorem{mythl}{\hei{引理}}[section]
\newtheorem{mytht}{\hei{定理}}[section]
\newtheorem{mythr}{\hei{注}}[section]
\newtheorem{mythc}{\hei{推论}}[section]
\newtheorem{mythd}{\hei{定义}}[section]
\newtheorem{mytha}{\hei{公理}}
\newtheorem{mythp}{\hei{命题}}
\newtheorem{mythe}{\hei{练习}}
\newtheorem{myli}{\hei{例}}[section]